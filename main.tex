\documentclass{article}
\usepackage{graphicx} % Required for inserting images


\title{\textbf{MATRIX COMPUTATIONS: HOMEWORK 2}}  % Declares the document's title.
\author{Andre Theo,
Dejean Maxime,
Doat Damien,
Nouidei Safiya}  
\date{October 2024}

\begin{document}

\maketitle

\section{Exercise A: Krylov subspaces}
\subsection*{(A1)}
As assumed in the assignement, $K_{r+1}(A,b) \subseteq K_{r}(A,b) $. 
If we can prove that we also have $K_{r+1}(A,b) \supseteq K_{r}(A,b) $, it means that we proved that $K_{r+1}(A,b) = K_{r}(A,b) $. 
Since, by definition of the krylov subspaces as $span\{b, Ab, ..., A^rb\}$,  all elements of $K_{r}(A,b)$ are in $K_{r+1}(A,b)$, we know that, indeed, $K_{r+1}(A,b) \supseteq K_{r}(A,b) $. And thus, we have proved the equality between the two spaces. 
We then need to show that, for $s\geq r$, $K_{s}(A,b) = K_{r}(A,b) $. It can be done with a proof by induction. 
Indeed, since the eqaulity holds for all $r$, it means that $K_{r+2}(A,b) = K_{r+1}(A,b) = K_{r}(A,b) $ etc. This means that for all $i$ such that $s-r\leq i$, $K_{r+i}(A,b) = K_{r+(i-1)}(A,b) =... =  K_{r}(A,b) $. 
This shows that $K_{s}(A,b) = K_{r}(A,b) $.


\subsection*{(A2)}
Let's prove that $ dim( K_r (A,b)) = r$ by induction. Since $ dim( K_n (A,b)) = s$ and $r \leq s$, it means that, 
at each iteration i so that $i \leq r$, the dimension of  $ dim( K_i (A,b)) = i$. Indeed, no combination will already be in the span
since r is smaller than s. 
We thus have:
\begin{itemize}
    \item $i = 1$: $ K_i (A,b) = \{b\}$ and $dim( K_i (A,b)) = 1$
    \item $i = 2$: $ K_i (A,b) = \{b, Ab\}$ and $dim( K_i (A,b)) = 2$
    \item \dots
    \item $i = r$: $ K_i (A,b) = K_r (A,b) $ and $dim( K_i (A,b)) = r$
\end{itemize}

\section{Exercise B: Arnoldi’s iteration}
\subsection*{(B1)}
We want to show that $ R_s $ has non-zero elements on its diagonal.
Fisrt, since $ K_s(A, b) = [b, Ab, \dots, A^{s-1}b] $, it consists of $ s $ linearly independent columns. Thus, $ K_s(A, b) $ has full column rank,
 so $ rank(K_s(A, b)) = s $.
In any QR decomposition $ M = QR$ where $ M \in R^{m \times s} $ and $ M $ has full column rank, R is invertible. 
Therefore, R is nonsingular, meaning all its diagonal elements are non-zero.
Finally, in our case, since $ K_s(A, b) $ has full column rank, $R_s $ is invertible and hence has non-zero elements on its diagonal.
Thus, we conclude that $ R_s $ has non-zero elements on its diagonal otherwise $dim(K_s (A,b)) \neq s$.

\subsection*{(B2)}
The goal is to show that for each $ 1 \leq r \leq s - 1 $,
\[
A Q_s[1:r] R_s[1:r, 1:r] = Q_s[1:r'] R_s[1:r', 2:r'],
\]
where $ r' = r + 1 $.
Let's do it step by step : 
\begin{enumerate}
    \item \textbf{Krylov Subspace Properties:} \\
    The Krylov subspace $ K_r(A, b) $ is defined as:
    \[
    K_r(A, b) = span\{b, Ab, A^2b, \dots, A^{r-1}b\}.
    \]
    By construction, applying $ A $ to $ K_r(A, b) $ generates $ K_{r+1}(A, b) $:
    \[
    K_{r+1}(A, b) = span\{b, Ab, A^2b, \dots, A^r b\}.
    \]
    Therefore, $ K_{r+1}(A, b) $ extends $ K_r(A, b) $ by including the new vector $ A^r b $, making it a shifted version of $ K_r(A, b) $ with an additional dimension.

    \item \textbf{QR Decomposition of $ K_r(A, b) $:} \\
    Given the QR decomposition of $ K_s(A, b) $, we have:
    \[
    K_s(A, b) = Q_s R_s,
    \]
    where $ Q_s \in R^{n \times s} $ is an orthonormal basis for $ K_s(A, b) $ and $ R_s \in R^{s \times s} $ is upper-triangular. The submatrices $ Q_s[1:r] $ and $ R_s[1:r, 1:r] $ provide a basis and the associated coefficients for the subspace $ K_r(A, b) $.

    \item \textbf{Applying $ A $ to $ Q_s[1:r] R_s[1:r, 1:r] $:} \\
    When we apply $ A $ to $ Q_s[1:r] R_s[1:r, 1:r] $, we obtain a matrix that spans the extended subspace $ K_{r+1}(A, b) $. This new subspace can be represented by:
    \[
    A Q_s[1:r] R_s[1:r, 1:r] = Q_s[1:r'] R_s[1:r', 2:r'],
    \]
    where $ r' = r + 1 $. 
\end{enumerate}


\subsection*{(B3)}


\subsection*{(B4)}


\subsection*{(B5)}


\subsection*{(B6)}


\subsection*{(B7)}

\newpage
\section{Exercise C: GMRES for linear system solution approximation}
\subsection*{(C1)}
\subsection*{(C2)}
\subsection*{(C3)}
\subsection*{(C4)}
\subsection*{(C5)}

\section{Exercise D: Arnoldi’s method for eigenvalue approximation}
\subsection*{(D1)}

\section{Exercise E: Implementation}

\end{document}